% Define document class
\documentclass[twocolumn]{aastex631}
\usepackage{showyourwork}
\usepackage{booktabs}


% Begin!
\begin{document}

% Title
\title{An open source scientific article}

% Author list
\author{Xueqing Chen}

% Abstract with filler text
\begin{abstract}
    Lorem ipsum dolor sit amet, consectetuer adipiscing elit.
    Ut purus elit, vestibulum ut, placerat ac, adipiscing vitae, felis.
    Curabitur dictum gravida mauris, consectetuer id, vulputate a, magna.
    Donec vehicula augue eu neque, morbi tristique senectus et netus et.
    Mauris ut leo, cras viverra metus rhoncus sem, nulla et lectus vestibulum.
    Phasellus eu tellus sit amet tortor gravida placerat.
    Integer sapien est, iaculis in, pretium quis, viverra ac, nunc.
    Praesent eget sem vel leo ultrices bibendum.
    Aenean faucibus, morbi dolor nulla, malesuada eu, pulvinar at, mollis ac.
    Curabitur auctor semper nulla donec varius orci eget risus.
    Duis nibh mi, congue eu, accumsan eleifend, sagittis quis, diam.
    Duis eget orci sit amet orci dignissim rutrum.
\end{abstract}

% Main body with filler text
\section{Introduction}
\label{sec:intro}

Lorem ipsum dolor sit amet, consectetuer adipiscing elit.
Ut purus elit, vestibulum ut, placerat ac, adipiscing vitae, felis.
Curabitur dictum gravida mauris, consectetuer id, vulputate a, magna.
Donec vehicula augue eu neque, morbi tristique senectus et netus et.
Mauris ut leo, cras viverra metus rhoncus sem, nulla et lectus vestibulum.
Phasellus eu tellus sit amet tortor gravida placerat.
Integer sapien est, iaculis in, pretium quis, viverra ac, nunc.
Praesent eget sem vel leo ultrices bibendum.
Aenean faucibus, morbi dolor nulla, malesuada eu, pulvinar at, mollis ac.
Curabitur auctor semper nulla donec varius orci eget risus.
Duis nibh mi, congue eu, accumsan eleifend, sagittis quis, diam.
Duis eget orci sit amet orci dignissim rutrum.

Nam dui ligula, fringilla a, euismod sodales, sollici- tudin vel, wisi.
Morbi auctor lorem non justo, nam lacus libero, pretium at, lobortis vitae.
Donec aliquet, tortor sed accumsan bibendum, erat ligula aliquet magna.
Morbi ac orci et nisl hendrerit mollis, suspendisse ut massa, cras nec ante.
Pellentesque a nulla cum sociis natoque penatibus et magnis dis parturient.
Aliquam tincidunt urna, nulla ullamcorper vestibulum turpis.
Pellentesque cursus luctus mauris

\section{Observations and Data Reduction}

\section{Doppler imaging}

\subsection{Model fitting}

Fit results:
\begin{table*}[ht!]
    \script{fit.py}
    \centering
    \caption{Fit results}
    \label{tab:my_table}
    %\variable{output/fit_table.txt}
\end{table*}

\iffalse
% figure of obs vs fitted spectra
\begin{figure*}[ht!]
    \script{fit.py}
    \begin{centering}
        \includegraphics[width=1\linewidth]{figures/fit.pdf}
        \caption{
            Plot showing a bunch of random numbers.
        }
        \label{fig:fit}
    \end{centering}
\end{figure*}

% figure of fitted vsini and rv
\begin{figure*}[ht!]
    \script{fit.py}
    \begin{centering}
    \begin{minipage}[b]{0.48\textwidth}
        \centering
        \includegraphics[width=\textwidth]{figures/fit_vsini.pdf}
        %\caption{Plot: \variable{output/fit_vsini.txt}}
        \label{fig:fit_vsini}
    \end{minipage}
    \hfill
    \begin{minipage}[b]{0.48\textwidth}
        \centering
        \includegraphics[width=\textwidth]{figures/fit_rv.pdf}
        %\caption{Plot: \variable{output/fit_rv.txt}}
        \label{fig:fit_rv}
    \end{minipage}
    \end{centering}
\end{figure*}
\fi

\subsection{Least-Square deconvolution}

\subsection{Maximum entropy Doppler imaging}

\subsection{Starry Doppler imaging}


\section{Doppler Maps}

\subsection{IGRINS K and H band maps}

\iffalse
% figure of IGRINS K maps
\begin{figure*}
    \script{doppler_imaging.py}
    \centering
    \begin{minipage}[b]{0.18\textwidth}
        \centering
        \includegraphics[width=\textwidth]{figures/igrinsK/solver1.pdf}
        %\caption{IC14new}
        \label{fig:igrinsKsolver1}
     \end{minipage}
     \hfill
     \begin{minipage}[b]{0.18\textwidth}
         \centering
         \includegraphics[width=\textwidth]{figures/igrinsK/solver2.pdf}
         %\caption{LSD+starry lin}
         \label{fig:igrinsKsolver2}
     \end{minipage}
     \hfill
     \begin{minipage}[b]{0.18\textwidth}
         \centering
         \includegraphics[width=\textwidth]{figures/igrinsK/solver3.pdf}
         %\caption{LSD+starry opt}
         \label{fig:igrinsKsolver3}
     \end{minipage}
     \hfill
     \begin{minipage}[b]{0.18\textwidth}
         \centering
         \includegraphics[width=\textwidth]{figures/igrinsK/solver4.pdf}
         %\caption{starry lin}
         \label{fig:igrinsKsolver4}
     \end{minipage}
     \hfill
     \begin{minipage}[b]{0.18\textwidth}
         \centering
         \includegraphics[width=\textwidth]{figures/igrinsK/solver5.pdf}
         %\caption{starry opt}
         \label{fig:igrinsKsolver5}
     \end{minipage}
\end{figure*}
\fi

\iftrue
% figure of IGIRNS H map
\begin{figure*}
    \script{run_igrinsH.py}
    \centering
    \begin{minipage}[b]{0.18\textwidth}
        \centering
        \includegraphics[width=\textwidth]{figures/igrinsH/solver1.pdf}
        %\caption{IC14new}
        \label{fig:igrinsHsolver1}
     \end{minipage}
     \hfill
     \begin{minipage}[b]{0.18\textwidth}
         \centering
         \includegraphics[width=\textwidth]{figures/igrinsH/solver2.pdf}
         %\caption{LSD+starry lin}
         \label{fig:igrinsHsolver2}
     \end{minipage}
     \hfill
     \begin{minipage}[b]{0.18\textwidth}
         \centering
         \includegraphics[width=\textwidth]{figures/igrinsH/solver3.pdf}
         %\caption{LSD+starry opt}
         \label{fig:igrinsHsolver3}
     \end{minipage}
     \hfill
     \begin{minipage}[b]{0.18\textwidth}
         \centering
         \includegraphics[width=\textwidth]{figures/igrinsH/solver4.pdf}
         %\caption{starry lin}
         \label{fig:igrinsHsolver4}
     \end{minipage}
     \hfill
     \begin{minipage}[b]{0.18\textwidth}
         \centering
         \includegraphics[width=\textwidth]{figures/igrinsH/solver5.pdf}
         %\caption{starry opt}
         \label{fig:igrinsHsolver5}
     \end{minipage}
\end{figure*} 
\fi

\section{Simulations}

\subsection{spot vs band model}

\iffalse
\begin{figure*}[ht!]
    \script{simulation.py}
    \centering
    \caption{simulations}
    \begin{minipage}[b]{0.2\textwidth}
        \centering
        \includegraphics[width=\textwidth]{figures/sim/fakemap.pdf}
        \label{fig:sim-fakemap}
    \end{minipage}
    \hfill
    \begin{minipage}[b]{0.2\textwidth}
        \centering
        \includegraphics[width=\textwidth]{figures/sim/tvplot.pdf}
        \label{fig:sim-tvplot}
    \end{minipage}
    \hfill
    \begin{minipage}[b]{0.2\textwidth}
        \centering
        \includegraphics[width=\textwidth]{figures/sim/tsplot.pdf}
        \label{fig:sim-tvplot}
    \end{minipage}
    \hfill\hfill\hfill\hfill\hfill\hfill\hfill
    \begin{minipage}[b]{0.2\textwidth}
    \end{minipage}
    
    \vfill
    
    \begin{minipage}[b]{0.18\textwidth}
        \centering
        \includegraphics[width=\textwidth]{figures/sim/solver1.pdf}
        %\caption{IC14new}
        \label{fig:sim-solver1}
    \end{minipage}
    \hfill
    \begin{minipage}[b]{0.18\textwidth}
        \centering
        \includegraphics[width=\textwidth]{figures/sim/solver2.pdf}
        %\caption{LSD+starry lin}
        \label{fig:sim-solver2}
    \end{minipage}
    \hfill
    \begin{minipage}[b]{0.18\textwidth}
        \centering
        \includegraphics[width=\textwidth]{figures/sim/solver3.pdf}
        %\caption{LSD+starry opt}
        \label{fig:sim-solver3}
    \end{minipage}
    \hfill
    \begin{minipage}[b]{0.18\textwidth}
        \centering
        \includegraphics[width=\textwidth]{figures/sim/solver4.pdf}
        %\caption{starry lin}
        \label{fig:sim-solver4}
    \end{minipage}
    \hfill
    \begin{minipage}[b]{0.18\textwidth}
        \centering
        \includegraphics[width=\textwidth]{figures/sim/solver5.pdf}
        %\caption{starry opt}
        \label{fig:sim-solver5}
    \end{minipage}
\end{figure*} 
\fi

%\variable{output/testout.txt}

\bibliography{bib}

\end{document}
